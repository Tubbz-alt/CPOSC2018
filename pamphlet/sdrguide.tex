\documentclass[10pt,foldmark,notumble]{leaflet}
\renewcommand*\foldmarkrule{.3mm}
\renewcommand*\foldmarklength{5mm}

%%% PAGE LAYOUT
%% Portrait
%% Short-Edge (Flip)

\usepackage{amsmath}
\usepackage[T1]{fontenc}
\usepackage{textcomp}
\usepackage{mathptmx}
\usepackage{xcolor}
\usepackage[scaled=0.9]{helvet}
\makeatletter
\def\ptmTeX{T\kern-.1667em\lower.5ex\hbox{E}\kern-.075emX\@}
\DeclareRobustCommand{\ptmLaTeX}{L\kern-.3em
        {\setbox0\hbox{T}%
         %\vb@xt@ % :-)
         \vbox to\ht0{\hbox{%
                            \csname S@\f@size\endcsname
                            \fontsize\sf@size\z@
                            \math@fontsfalse\selectfont
                            A}%
                      \vss}%
        }%
        \kern-.12em
        \ptmTeX}
\makeatother
\let\TeX=\ptmTeX
\let\LaTeX=\ptmLaTeX
\usepackage{shortvrb}
\MakeShortVerb{\|}
\usepackage{url}
\usepackage{graphicx}
%\usepackage{lipsum}
%\usepackage{blindtext}

%%%%\renewcommand{\descfont}{\normalfont}
\newcommand\Lpack[1]{\textsf{#1}}
\newcommand\Lclass[1]{\textsf{#1}}
\newcommand\Lopt[1]{\texttt{#1}}
\newcommand\Lprog[1]{\textit{#1}}
\definecolor{LIGHTGRAY}{gray}{.8}

\newcommand*\defaultmarker{\textsuperscript\textasteriskcentered}


\title{RTL-SDR Getting Started Guide}

\author{Central Pennsylvania Open Source Conference}
\date{{\scriptsize \textbf{Presented by:} Tom Swartz \hfill \textbf{tom@tswartz.net}}}

\CutLine*{1}% Dotted line without scissors
\CutLine*{6}% Remove * for scissors

%\AddToBackground{1}{%  Background of a small page
%  \put(0,0){\textcolor{Cerulean}{\rule{\paperwidth}{\paperheight}}}}


\AddToBackground{1}{%  Background of a small page
  \put(115,530){\includegraphics[scale=0.15]{images/logo.png}}}


%\AddToBackground{6}{%  Background of a small page
%  \put(0,0){\textcolor{YellowOrange}{\rule{\paperwidth}{\paperheight}}}}


\AddToBackground*{2}{% Background of a large page
  \put(\LenToUnit{.5\paperwidth},\LenToUnit{.5\paperheight}){%
    \makebox(0,0)[c]{%
      \resizebox{.9\paperwidth}{!}{\rotatebox{35.26}{%
        \textsf{\textbf{\textcolor{LIGHTGRAY}{CPOSC 2018}}}}}}}}


\begin{document}
\maketitle
\thispagestyle{empty}
\section{Quick Start\color{red}\hrulefill\color{black}}
\textbf{Items Included With This Kit}\\
Included in this kit are several items to help get started right away with
software defined radio.
This pamphlet outlines what each of these items do, how to use them,
and highlights some interesting things you can receive with the kit.
The next page reviews some beginner questions about SDR\@.
\begin{itemize}
        \item Software Defined Radio
        \item Dipole antenna base, with two pairs of antennas
        \item SMA connector antenna extension cable
        \item Flexible tripod mount
        \item Suction cup window mount
        \item Custom-made 140 MHz Band-Pass Filter
\end{itemize}

\textbf{Installing Software}\\
The software you use will depend on your computer operating system. Later
sections in this pamphlet outline specific programs for each Operating System.
In general, you need to install a program to interpret the SDR device's
output.

By and large, most SDR applications run on Linux (and Raspberry
Pi), so Linux is generally recommended.
Microsoft Windows is also well supported, but typically requires driver
installation prior to use.

See \url{https://www.rtl-sdr.com/QSG} for more information specific to your
computer platform.

\textbf{Connecting the Hardware}\\
Plug the software defined radio into a USB port.
Connect the antenna to the SMA port on the software defined radio.
Place the antenna outside, or as close to an outside window as possible.

\section{Common Software Defined Radio Questions\color{red}\hrulefill\color{black}}
\textbf{What is Software Defined Radio?}\\
Radio components such as modulators, demodulators, and tuners are traditionally
implemented in hardware components. The advent of modern computing and
analog-to-digital converters allows most of these traditionally hardware based
components to be implemented into software instead. Hence, the term software
defined radio. This enables easy signal processing on inexpensive radio hardware
to be used at a large scale.

\textbf{What can I do with Software Defined Radio?}\\
SDRs like the one included in this kit can receive signals from 24 MHz to 2 GHz.
This typically means that it can be used to listen to a variety of transmissions,
either from local police radios, to EMS/Ambulances, to regular FM Radio like you
receive in your car.
Other common signals can be air traffic control, weather balloons, satellites,
and amateur radio transmissions.
Digital signals can also be received, so long as they are within the frequency
range of the software defined radio.

\textbf{Can I sniff or receive WiFi with an SDR?}\\
No. These devices can only receive signals from 24 MHz to 2 GHz, in 2 MHz
`slices'.
Typical WiFi signals are found at 2.4 GHz and higher, and operate
in 22 MHz bandwidth `slices'; much too wide for this type of device.

\section{About the Band Pass Filter\color{red}\hrulefill\color{black}}
\textbf{The Purpose of a Filter}\\
Because the SDR devices can receive such a wide range of signals, it's not
uncommon for strong nearby transmissions to `overload' or `shout over' a faint,
faraway signal.

A filter will limit the range of frequencies that the SDR can hear through the
antenna, and therefore improve the \emph{signal to noise ratio}.
Any transmissions on frequencies not within that range are `dampened'; their
strength is greatly reduced.
This means a certain band of frequencies can pass through the filter easily,
leading to the name \emph{Band Pass Filter}.

It is entirely possible that very strong local transmissions outside of the band
pass filter range can still be received by the SDR, but these transmissions will
be significantly lower in strength (quieter) than if they were unfiltered.

\textbf{Using the Band Pass Filter}\\
The filter included in this kit only allows signals from 130 MHz to 150 MHz
to be heard clearly by the software defined radio.
The band pass filter was designed specifically to filter out any frequencies not
used by common weather and radio satellites.
It helps reduce outside noise and ensures that the clearest signal possible is recorded.

To use the filter, simply connect it in-between the antenna and the SDR dongle,
taking care to have the antenna on the ``in'' side, and the SDR on the ``out'' side.
If you are using the extension cable provided in the kit, place the filter as
close to the antenna as possible for best results.

It is not recommended to use the filter in the SDR antenna system if you
are interested in receiving signals outside of the specified filter range.

\textbf{Open Source Hardware}\\
The band-pass filter included in this is open-source hardware and was designed
using KiCAD, an open-source circuit design program.

Because of the design and part selection, five filters can be built for less
than \$20.

Schematics for this device, as well as instructions to build the filter
can be found at:\\
\url{https://github.com/tomswartz07/140bpf-kicad}.

\section{Notable Linux SDR Applications\color{red}\hrulefill\color{black}}
\textbf{GQRX}\\
GQRX is the gold-standard for open-source SDR user interfaces. It supports a
wide variety of reception modes, and has the ability to stream SDR data over
the internet. Also works on OSX and Windows.

\textbf{rtl\_fm}\\
rtl\_fm is a command-line based receiver written by a Lancaster County native.
It is extremely useful as it can be used to pipe decoded audio to various
other decoder programs.

\textbf{dump1090}\\
dump1090 decodes aircraft location data, which is transmitted on
1090 MHz. This data is shown on a map, with markers indicating which airline
and flight number is being received.

\textbf{WXtoIMG}\\
WXtoIMG is a program written in the late 1990's to convert NOAA Weather satellite
data into images. It supports a variety of decoding options, and can overlay
maps and other useful data on top of the image.
Unfortunately, it is very very particular about the way a signal was recorded.

\textbf{Multimon-ng}\\
Multimon-ng is an application to decode digital transmissions into text.
The program can decode pager messages (more common than you would think!), as
well as Emergency Alerts, Morse Code, and a variety of others.

\textbf{GPredict}\\
GPredict is an open source satellite prediction software, written by the same
author of GQRX\@. It has very tight integration with GQRX software, and, most
notably is used by the European Space Agency to track satellites.

\textbf{GnuRadio Companion}\\
GnuRadio Companion (GRC) is an incredibly powerful `LEGO Block' style interface
for receiving and processing SDR signals. It allows for unlimited potential
for decoding various signals or creating your own user interface.

\textbf{rtl\_433}\\
rtl\_433 is a command line program to receive and decode outdoor temperature
sensors, garage door openers, wireless doorbells, and other common wireless
consumer products.

\section{Notable Windows SDR Applications\color{red}\hrulefill\color{black}}
\textbf{SDR\#}\\
`SDR Sharp' is an excellent, well supported SDR user interface that supports
a large variety of plugins.

\textbf{CubicSDR}\\
CubicSDR is a cross-platform SDR application. It's most notable feature
is a scanner-like feature, allowing the SDR to monitor multiple frequencies
at the same time.

\textbf{RDS Spy}\\
RDS Spy is a Radio Data System (RDS) decoder.
This application can decode song title and radio station information which
is broadcast along with some local FM music transmissions.

\section{Receiving Weather Satellite Data\color{red}\hrulefill\color{black}}
\textbf{Select a good reception location}\\
In order to get the best possible signal from a satellite,
you need an unobstructed view of the sky overhead.

It's important to find an outside location, such as a backyard, local park, or
nearby field, so that you can have very few objects between your antenna and the
satellite in orbit.

\textbf{Determine when the satellite will fly overhead}\\
Most weather satellites orbit around the Earth from the North Pole to the South
Pole. This typically means that the satellite will pass overhead twice
per day; once in the morning, once in the evening.

Use satellite prediction software (such as GPredict) or an online
resource (such as \url{https://www.n2yo.com}) to determine what times
the satellite will fly over your location.

\textbf{Receiving the Signal}\\
At the appropriate time, set up your SDR software,
using a \emph{Narrow FM} signal type with a very wide \emph{Filter Width}
(NOAA satellites use 80 k).
Verify that the \emph{Squelch} is set to the lowest possible value. You should
hear static through the computer speakers. Ensure the antenna is unobstructed.

Tune to the appropriate frequency for the satellite:
\begin{description}
\item[NOAA 15] \dotfill 137.620 MHz
\item[NOAA 18] \dotfill 137.912 MHz
\item[NOAA 19] \dotfill 137.100 MHz
\end{description}

Hit the \emph{Record} button to record a \emph{*.wav} file
of the signal as it passes overhead.

\textbf{Decoding the Image}\\
Most common NOAA decoder software expects \emph{Mono 11025 Hz WAV} files, so you
must use Audacity to process the audio file into the expected format prior to
decoding an image.

Typically, use \emph{Tracks} $\to$ \emph{Stereo Track to Mono}, then
select \emph{Effects} $\to$ \emph{Normalize} to standardize the audio level.
Ensure that the \emph{Project Rate} in Audacity is set to \emph{11025} and then
export the audio as a \emph{16 Bit PCM Microsoft WAV} file.

Finally, load the file into a decoder program, such as\\ WXtoIMG, to view the
weather satellite image!

\section{Interesting Lancaster-Area Frequencies\color{red}\hrulefill\color{black}}
\begin{description}
\item[033.440 MHz] \dotfill Lancaster County Emergency Services
\item[072.220 MHz] \dotfill Lancaster General Hospital
\item[125.675 MHz] \dotfill Lancaster Airport Weather
\item[137.100 MHz] \dotfill NOAA 19 Weather Satellite
\item[137.620 MHz] \dotfill NOAA 15 Weather Satellite
\item[137.912 MHz] \dotfill NOAA 18 Weather Satellite
\item[137.890 MHz] \dotfill Meteor M2 Weather Satellite
\item[145.800 MHz] \dotfill International Space Station
\item[162.550 MHz] \dotfill National Weather Service
\item[452.325 MHz] \dotfill Park City Mall Security
\end{description}

\emph{Many other signals are being transmitted at all hours of the day. Use the
SDR to scan around and see what you find!}

\section{Useful SDR Resources\color{red}\hrulefill\color{black}}
\begin{itemize}
\item \url{https://www.rtl-sdr.com/QSG}\\{\scriptsize Quick Start Guide for RTL-SDR devices}
\item \url{https://www.rtl-sdr.com/V3}\\{\scriptsize User guide for the SDR dongle included in this kit}
\item \url{https://www.rtl-sdr.com/DIPOLE}\\{\scriptsize Instructions specifically for antennas in this SDR kit}
\item \url{https://www.radioreference.com}\\{\scriptsize List frequencies for public works and business in your area}
\item \url{https://www.repeaterbook.com}\\{\scriptsize List common amateur radio (ham radio) frequencies}
\item \url{https://www.n2yo.com}\\{\scriptsize List satellite fly-overs in your area}
\end{itemize}

\section{Notes\color{red}\hrulefill\color{black}}

\end{document}
